\href{https://app.travis-ci.com/Ykulkarni-ops/enpm808x_tdd}{\tt } \section*{\href{https://coveralls.io/github/Ykulkarni-ops/enpm808x_tdd?branch=master}{\tt } }

\#\# Authors 
\begin{DoxyCode}
Part 1: 
* Navigator: Diane Ngo (dngo13)
* Driver: Ameya Konkar (ameyakonk)
@version 1.0
\end{DoxyCode}



\begin{DoxyCode}
Part 2: 
* Navigator: Aditi Ramadwar
* Driver: Yash Kulkarni
@version 1.0
\end{DoxyCode}
 \subsection*{Overview}

A Test-\/\+Driven Development assignment C++ project for a P\+ID controller with\+:


\begin{DoxyItemize}
\item cmake
\item googletest
\item Travis CI
\item Coveralls
\end{DoxyItemize}

\subsection*{To do instructions}


\begin{DoxyItemize}
\item The implementation of the part 2 has been completed and the unit test cases have been verified successfully.
\item Extra test case was added to ensure the code coverage.
\item C++ Google style guide is verified through cpplint and cppcheck.
\item Doxygen style has been followed for commenting.
\item Unit test cases and code coverage has been verified using Travis-\/ci and Coveralls.\+io
\end{DoxyItemize}

\#\# Standard install via command-\/line 
\begin{DoxyCode}
git clone --recursive https://github.com/dngo13/enpm808x\_tdd
cd <path to repository>
mkdir build
cd build
cmake ..
make
Run tests: ./test/cpp-test
Run program: ./app/shell-app
\end{DoxyCode}


\#\# Building for code coverage (for assignments beginning in Week 4) 
\begin{DoxyCode}
sudo apt-get install lcov
cmake -D COVERAGE=ON -D CMAKE\_BUILD\_TYPE=Debug ../
make
make code\_coverage
\end{DoxyCode}
 This generates a index.\+html page in the build/coverage sub-\/directory that can be viewed locally in a web browser.

\subsection*{Working with Eclipse I\+DE}

\subsection*{Installation}

In your Eclipse workspace directory (or create a new one), checkout the repo (and submodules) 
\begin{DoxyCode}
mkdir -p ~/workspace
cd ~/workspace
git clone --recursive https://github.com/dpiet/cpp-boilerplate
\end{DoxyCode}


In your work directory, use cmake to create an Eclipse project for an \mbox{[}out-\/of-\/source build\mbox{]} of cpp-\/boilerplate


\begin{DoxyCode}
cd ~/workspace
mkdir -p boilerplate-eclipse
cd boilerplate-eclipse
cmake -G "Eclipse CDT4 - Unix Makefiles" -D CMAKE\_BUILD\_TYPE=Debug -D CMAKE\_ECLIPSE\_VERSION=4.7.0 -D
       CMAKE\_CXX\_COMPILER\_ARG1=-std=c++14 ../cpp-boilerplate/
\end{DoxyCode}


\subsection*{Import}

Open Eclipse, go to File -\/$>$ Import -\/$>$ General -\/$>$ Existing Projects into Workspace -\/$>$ Select \char`\"{}boilerplate-\/eclipse\char`\"{} directory created previously as root directory -\/$>$ Finish

\section*{Edit}

Source files may be edited under the \char`\"{}\mbox{[}\+Source Directory\mbox{]}\char`\"{} label in the Project Explorer.

\subsection*{Build}

To build the project, in Eclipse, unfold boilerplate-\/eclipse project in Project Explorer, unfold Build Targets, double click on \char`\"{}all\char`\"{} to build all projects.

\subsection*{Run}


\begin{DoxyEnumerate}
\item In Eclipse, right click on the boilerplate-\/eclipse in Project Explorer, select Run As -\/$>$ Local C/\+C++ Application
\item Choose the binaries to run (e.\+g. shell-\/app, cpp-\/test for unit testing)
\end{DoxyEnumerate}

\subsection*{Debug}


\begin{DoxyEnumerate}
\item Set breakpoint in source file (i.\+e. double click in the left margin on the line you want the program to break).
\item In Eclipse, right click on the boilerplate-\/eclipse in Project Explorer, select Debug As -\/$>$ Local C/\+C++ Application, choose the binaries to run (e.\+g. shell-\/app).
\item If prompt to \char`\"{}\+Confirm Perspective Switch\char`\"{}, select yes.
\item Program will break at the breakpoint you set.
\item Press Step Into (F5), Step Over (F6), Step Return (F7) to step/debug your program.
\item Right click on the variable in editor to add watch expression to watch the variable in debugger window.
\item Press Terminate icon to terminate debugging and press C/\+C++ icon to switch back to C/\+C++ perspetive view (or Windows-\/$>$Perspective-\/$>$Open Perspective-\/$>$C/\+C++).
\end{DoxyEnumerate}

\subsection*{Plugins}


\begin{DoxyItemize}
\item Cpp\+Ch\+Eclipse

To install and run cppcheck in Eclipse
\begin{DoxyEnumerate}
\item In Eclipse, go to Window -\/$>$ Preferences -\/$>$ C/\+C++ -\/$>$ cppcheclipse. Set cppcheck binary path to \char`\"{}/usr/bin/cppcheck\char`\"{}.
\item To run C\+P\+P\+Check on a project, right click on the project name in the Project Explorer and choose cppcheck -\/$>$ Run cppcheck.
\end{DoxyEnumerate}
\item Google C++ Sytle

To include and use Google C++ Style formatter in Eclipse
\begin{DoxyEnumerate}
\item In Eclipse, go to Window -\/$>$ Preferences -\/$>$ C/\+C++ -\/$>$ Code Style -\/$>$ Formatter. Import \href{https://raw.githubusercontent.com/google/styleguide/gh-pages/eclipse-cpp-google-style.xml}{\tt eclipse-\/cpp-\/google-\/style} and apply.
\item To use Google C++ style formatter, right click on the source code or folder in Project Explorer and choose Source -\/$>$ Format
\end{DoxyEnumerate}
\item Git

It is possible to manage version control through Eclipse and the git plugin, but it typically requires creating another project. If you\textquotesingle{}re interested in this, try it out yourself and contact me on Canvas. 
\end{DoxyItemize}